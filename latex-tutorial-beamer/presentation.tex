\documentclass{beamer}							%ορισμός του documentclass που είναι για τη δημιουργία παρουσιάσεων
\usepackage{fontspec}
\usepackage{xltxtra}
\usepackage{xgreek}
\usepackage{graphicx}
\usetheme{PaloAlto}								%Θέμα που χρησιμοποιούμε για την εμφάνιση της παρουσίασης
\usepackage{hyperref}
\logo{\includegraphics[height=1.48cm]{ubuntu}}  %Εισαγωγή λογότυπου πάνω αριστερά
\setsansfont{GFS Didot}							%Ορισμός γραμματοσειράς
\usenavigationsymbolstemplate{}					%Αφαιρούμε τα κουμπιά κάτω δεξιά
\title{Μαθε παιδι μου Linux (και λιγο \texorpdfstring\XeLaTeX)) }
\author{\href{www.Cerebrux.net}{Cerebrux.net}}
\date{Ιούνιος 2016}

\begin{document}
\maketitle

\begin{frame}{Εισαγωγη}
\begin{abstract}
Εισαγωγή, ευχαριστίες και λίγα λογία για την παρουσίαση μας σε \XeLaTeX
\end{abstract}
\end{frame}

\begin{frame}{Διάρθρωση παρουσίασης}
\tableofcontents
\end{frame}

\begin{frame}{Λίγα πράγματα για το linux}
\section{Τι είναι το Ubuntu}
\begin{center}
\begin{definition}
Μια από τις δημοφιλέστερες διανομές Linux είναι το Ubuntu

\end{definition}

\includegraphics[width=0.7\linewidth]{ubuntu-logo}
\end{center}
\end{frame} 

\begin{frame}{Μπλα μπλα...}
\section{Μέρος της παρουσίασης}
\begin{alertblock}{μπλα...}
 Μπλα μπλα...
\end{alertblock}
\end{frame} 

\begin{frame}{Μπλα μπλα...}
\section{Μέρος της παρουσίασης}
\begin{itemize}
\item Item 1
\item Item 2
\item Item 3
\end{itemize}

\end{frame} 

\begin{frame}{Τέλος της παρουσίασης}
\section{Τέλος της παρουσίασης}
Και εδώ τελειώσαμε
\begin{center}
{\LARGE Ευχαριστούμε για την προσοχή σας}\\
\vspace{24pt}
\url {www.cerebrux.net}
\end{center}
\end{frame}

\end{document}