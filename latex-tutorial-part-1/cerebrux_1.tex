\documentclass[a4paper,titlepage,oneside,12pt]{article}
    \usepackage{fontspec}
    \usepackage{xunicode}
    \usepackage{xltxtra}
    \usepackage{xgreek}
    \usepackage{graphicx} %πακέτο για εισαγωγή γραφικών
    \usepackage{hyperref} %πακέτο για εισαγωγή υπερσυνδέσμων
    \setmainfont[Mapping=tex-text]{Ubuntu}


\title{Cerebrux - Μάθε παιδί μου Linux}
\author{Εσείς}
%\date{Η ημερομηνία που θέλουμε} %κάνοντας uncomment ορίζουμε την ημερομηνία που θέλουμε
  
    
\begin{document}
\maketitle  

\section{Το Λειτουργικό Σύστημα Linux} 

\begin{figure} [!hbp]
\centering
% Εξασφαλίζουμε το μέγιστο μέγεθος για την εικόνα, χωρίς να χαλάμε τις
% αναλογίες της. Το αρχείο πρέπει να βρίσκεται στο ίδιο directory
% με το αρχείο μας.
\includegraphics[width=1\linewidth]{linux-logo.png} 
\caption{Λογότυπο του Linux}
\end{figure}

\subsection{Ορισμός}
\textbf{Τι είναι το Linux;} 
\medskip

\textit{Το Linux (Λίνουξ) ή GNU/Linux (Γκνού/Λίνουξ),είναι ένα λειτουργικό σύστημα που αποτελείται από ελεύθερο λογισμικό. Η χρήση του είναι παρόμοια με αυτή του Unix, αλλά όλος ο πηγαίος κώδικας του έχει γραφτεί από την αρχή ως ελεύθερο λογισμικό υπό την ελεύθερη άδεια χρήσης GNU General Public License\footnote{\url{https://el.wikipedia.org/wiki/Linux}}}
\end{document}