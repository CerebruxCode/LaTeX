\documentclass[a4paper,12pt]{article}
    \usepackage{fontspec}
    \usepackage{xunicode}
    \usepackage{xltxtra}
    \usepackage{xgreek}
    \setmainfont[Mapping=tex-text]{FreeSans}
    \usepackage{tikz}
    
\begin{document}
Στο παρόν άρθρο θα ασχοληθούμε με τη δημιουργία γραφημάτων με τη χρήση του tikz
\vspace{10pt}

\begin{tikzpicture}
\draw (5,0) -- (-5, 0);
\end{tikzpicture}

\vspace{10pt}
Πιο σύνθετα μπορούμε να κάνουμε\\
\vspace{10pt}

\begin{tikzpicture}
\draw   (5,0) -- (-5, 0)
        (0,2) -- (0,-2);
\end{tikzpicture}

\vspace{10pt}
Πιο σύνθετα μπορούμε να κάνουμε\\
\vspace{10pt}

\begin{tikzpicture}
\draw[green, dotted, ultra thick, rotate=15]   
        (5,0) -- (-5, 0)
        (0,2) -- (0,-2);
\end{tikzpicture}

\begin{tikzpicture}
\draw[->](0,0) -- (12,0);
\end{tikzpicture}

\begin{tikzpicture}
\draw[<<->>](0,0) -- (12,0);
\end{tikzpicture}

\begin{tikzpicture}
\draw[<<->>, ultra thick, blue](0,0) -- (12,0);
\end{tikzpicture}

\begin{tikzpicture}
\draw[|->>, thick, red](0,0) -- (12,0);
\end{tikzpicture}


\begin{tikzpicture}
\draw (0,0) circle (3cm);
\draw (0,0) ellipse (3cm and 1cm);
\draw (3,1) arc (0:75:4cm);
\end{tikzpicture}

\begin{tikzpicture}
\draw[green,thick,dashed] (2,2) circle (3cm);
\end{tikzpicture}


\end{document}

