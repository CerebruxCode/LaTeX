\documentclass[a4paper,titlepage,oneside,12pt]{article}
    \usepackage{fontspec}
    \usepackage{xunicode}
    \usepackage{xltxtra}
    \usepackage{xgreek}
    \usepackage{graphicx} %πακέτο για εισαγωγή γραφικών
    \usepackage{hyperref} %πακέτο για εισαγωγή υπερσυνδέσμων
    \usepackage{fancyhdr} %πακέτο για επικεφαλίδες στις σελίδες
    \usepackage{multicol} %πακέτο για πολλαπλές στήλες στο κείμενο
    \usepackage{amsmath}
    \setlength{\columnsep}{1cm} % ορίζουμε την απόσταση μεταξύ των στηλών
    \parindent=0in  % Ενεργοποιήστε την ακόλουθη γραμμή αν δεν θέλετε στοίχιση στις νέες παραγράφους.
     
    \setmainfont[Mapping=tex-text]{GFS Didot}
    \setlength{\headheight}{15pt}
    \renewcommand{\figurename}{Εικόνα} %μετονομασία του υποτίτλου της εικόνας
 
 
\title{Cerebrux - Μάθε παιδί μου Linux}
\author{Οι αναγνώστες μας}
\pagestyle{fancy}
 
   
     
\begin{document}
 
 
\section{Το Λειτουργικό Σύστημα Linux} 
 
\begin{figure} [!hbp]
\centering
% Εξασφαλίζουμε το μέγιστο μέγεθος για την εικόνα, χωρίς να χαλάμε τις
% αναλογίες της. Το αρχείο πρέπει να βρίσκεται στο ίδιο directory
% με το αρχείο μας.
\includegraphics[width=1\linewidth]{linux-logo.png} 
\caption{Λογότυπο του Linux}
\end{figure}
 
\subsection{Ορισμός}
\textbf{Τι είναι το Linux;}
\medskip
 
\textit{Το Linux (Λίνουξ) ή GNU/Linux (Γκνού/Λίνουξ),είναι ένα λειτουργικό σύστημα που αποτελείται από ελεύθερο λογισμικό. Η χρήση του είναι παρόμοια με αυτή του Unix, αλλά όλος ο πηγαίος κώδικας του έχει γραφτεί από την αρχή ως ελεύθερο λογισμικό υπό την ελεύθερη άδεια χρήσης GNU General Public License\footnote{\url{https://el.wikipedia.org/wiki/Linux}}}\\
Στη συνέχεια του κειμένου μας θα δούμε πως θα κάνουμε ένα δίστηλο, αλλά και πως θα σταματήσουμε τη παρούσα σελίδα προκειμένου να ξεκινήσουμε καινούρια.
\pagebreak
 
\begin{multicols}{2}
[
\section{Πως κάνουμε δίστηλο}
Εδώ μπορούμε να πούμε λίγα λόγια προτού ξεκινήσουμε το δίστηλο μας]
Το Linux συχνά προσφέρεται στο χρήστη σε διάφορες διανομές Linux. Χαρακτηριστικό των διανομών είναι η μεγάλη δυνατότητα παραμετροποίησης και επιλογής που προσφέρουν καθώς κάθε μια απευθύνεται σε διαφορετικό τύπο χρηστών. Ανάλογα με την φιλοσοφία που ακολουθεί κάθε διανομή μπορεί να δίνει μεγαλύτερη βάση στη φιλικότητα προς τον χρήστη, στις εφαρμογές πολυμέσων, την ευκολία παραμετροποίησης, απλότητα του συστήματος, μόνο ελεύθερο λογισμικό, χαμηλές απαιτήσεις σε πόρους, και άλλα.
\end{multicols}
 
Δημιουργός του πυρήνα Linux είναι ο Λίνους Τόρβαλντς, από το όνομα του οποίου
προήλθε και η ονομασία Linux. O Τόρβαλντς άρχισε να αναπτύσσει ένα αρχικό πυρήνα το 1991
χρησιμοποιώντας κώδικα από από το ακαδημαϊκό λειτουργικό σύστημα MINIX του Άντριου 
Τάνενμπάουμ, το οποίο και μετεξέλιξε ανεξάρτητα, και κατόπιν υιοθέτησε τα προγράμματα και βιβλιοθήκες
του λειτουργικού συστήματος GNU του Ρίτσαρντ Στάλλμαν. Πάνω στον αρχικό πυρήνα του
Τόρβαλντς έχουν εργαστεί χιλιάδες χρήστες, κοινότητες αλλά και εταιρείες. Λόγω της
συνύπαρξης του πυρήνα Linux και του συστήματος GNU στο σχηματισμό του Linux ως 
λειτουργικό σύστημα, συχνά το σύστημα αυτό αναφέρεται ώς GNU/Linux, 
όπως προτιμά το Ίδρυμα Ελεύθερου 
Λογισμικού
 
\pagebreak
\section{Εισαγωγή μαθηματικών συμβόλων}
\parindent=0in
Θα δούμε το ανάπτυγμα της ταυτότητας $(a+b)^2$
\begin{equation}
(a+b)^2=a^2+2ab+b^2
\end{equation}
εν συνεχεία το ανάπτυγμα της $(a-b)^2$
\begin{equation}
(a-b)^2=a^2-2ab+b^2
\end{equation}
\begin{equation} 
    \alpha = \beta^2_i + \gamma^2_i 
\end{equation}
 
\begin{align} 
    \alpha = \beta\\
    \beta = \gamma
\end{align}
 
\end{document}