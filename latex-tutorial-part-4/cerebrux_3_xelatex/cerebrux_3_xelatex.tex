\documentclass[a4paper,titlepage,oneside,12pt]{article}
    \usepackage{fontspec}
    \usepackage{xunicode}
    \usepackage{xltxtra}
    \usepackage{xgreek}
    \usepackage{graphicx} %πακέτο για εισαγωγή γραφικών
    \usepackage{hyperref} %πακέτο για εισαγωγή υπερσυνδέσμων
    \renewcommand{\figurename}{Εικόνα} %μετονομασία του υποτίτλου της εικόνας
	\usepackage[nottoc]{tocbibind}     %Πακέτο για εμφάνιση βιβλιογραφίας στα περιεχόμενα
    \setmainfont{GFS Didot}
\begin{document}
\tableofcontents % εμφάνιση περιεχομένων
\newpage
\section{Τι είναι το Arch Linux}

\begin{figure}[hbp!]
\centering
\includegraphics[scale=1]{archlinux.png}
\caption{λογότυπο του Arch Linux}
\end{figure}

Το Arch Linux\cite{archlinux} είναι μια ανεξάρτητα ανεπτυγμένη i686/x86\_64 διανομή γενικής χρήσεως, βασισμένη στο μοντέλο αναβαθμίσεων rolling-release\cite{rolling} και αρκετά ευέλικτη ώστε να ταιριάζει σε οποιονδήποτε ρόλο. Η ανάπτυξή του εστιάζει στην απλοϊκότητα, στο μινιμαλισμό και τον ορθό κώδικα. Το Arch εγκαθίσταται ως ένα βασικό σύστημα και ρυθμίζεται από το χρήστη ώστε να χρησιμοποιεί το ιδανικό του γραφικό περιβάλλον και να εγκαθιστά μόνο ότι βρίσκει επιθυμητό για τους μοναδικούς του σκοπούς. 
\pagebreak

\subsection{Πλεονεκτήματα}
Το Arch έχει ένα ελαφρύ περιβάλλον μετά την εγκατάσταση, (περιβάλλον γραμμής εντολών), ήδη συνεταγμένο και βελτιστοποιημένο για i686/x86\_64 αρχιτεκτονικές. Είναι γρήγορο, ελαφρύ, ευέλικτο και απλό. Η φιλοσοφία και η εφαρμογή του σχεδιασμού του, το κάνουν εύκολο στο να επεκταθεί και να προσαρμοστεί σε οποιοδήποτε τύπο μηχανήματος και αν χτίζετε-από ένα μινιμαλιστικό μηχάνημα κονσόλας, στα πιο επιβλητικά και πλούσια περιβάλλοντα χρήστη, διαθέσιμα. 

\subsubsection{Διαχείριση πακέτων}
Το Arch υποστηρίζεται από ένα εύκολο στη χρήση σύστημα δυαδικών πακέτων, το pacman, που επιτρέπει στο χρήστη την αναβάθμιση ολόκληρου του συστήματός σας με μία εντολή. Ο Pacman είναι γραμμένος σε C και είναι ελαφρύς και γρήγορος. Το Arch επίσης χρησιμοποιεί ένα σύστημα παρόμοιο με τα ports για χτίσημο πακέτων (Arch Build System) για να κάνει εύκολη την κατασκευή πακέτων (compile), η οποία μπορεί επίσης να συγχρονιστεί με μία εντολή. Ολόκληρο το σύστημά μπορεί να ξανακτιστεί με μία εντολή. Το μοντέλο rolling release επιτρέπει στο σύστημα την συνεχή και αδιάκομη αναβάθμιση, χωρίς ανάγκη για επανεγκατάσταση.

Το Arch Linux προσπαθεί να προσφέρει την σταθερότερη καινούρια έκδοση για το λογισμικό του. Οι χρήστες υποστηρίζουμε ένα βελτιωμένο πυρήνα πακέτων που έχει οριστεί για το βασικό i686 και x86-64 σύστημα, χιλιάδες πρόσθετα, υψηλής ποιότητας δυαδικά πακέτα που βρίσκονται ανάμεσα στα αποθετήρια χρηστών (Arch User Repositories, AUR), συνοδευόμενα PKGBUILD scripts, για την κατασκευή (building) και τη συσκευασία (packaging) από την πηγή (source). Το Arch στοχέυει στη διανομή προγραμμάτων που δεν τροποποιήθηκαν (vanilla software), δηλαδή τα πακέτα προσφέρονται όπως τα διανέμει εξ αρχής ο προγραμματιστής τους. Διορθώσεις προκύπτουν μόνο σε σπάνιες περιπτώσεις, ώστε να αποφθευχθεί αστάθεια που μπορεί να προκύψει σε αυτό το μοντέλο αναβαθμίσεων. 

\begin{thebibliography}{2}

\bibitem{archlinux}Arch Linux: \emph{Latest Installation Medium}. 2013
\bibitem{rolling}Λίνους Τόρβαλντς: \emph{Ο Πυρήνας του Linux}, 1991
  
\end{thebibliography}

\end{document}