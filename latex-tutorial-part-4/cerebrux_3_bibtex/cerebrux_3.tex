\documentclass[a4paper,12pt]{article}

%\usepackage{ucs}                 %πακέτο για utf8 encoding
\usepackage[utf8]{inputenc}       %πακέτο για utf8 κωδικοποίηση 
\usepackage[greek,english]{babel} %ελληνικά με τη χρήση του πακέτου babel
\newcommand{\en}{\selectlanguage{english}} %Δημιουργία εντολής για εύκολη αλλαγή στα Ελληνικά
\newcommand{\gr}{\selectlanguage{greek}}   %Δημιουργία εντολής για εύκολη αλλαγή στα Αγγλικά
\usepackage{babelbib}                      %Πακέτο για βιβλιογραφία
\usepackage{csquotes}                      %Πακέτο για βιβλιογραφία (προτείνεται από τους δημιουργούς του biblatex
\usepackage[nottoc]{tocbibind}             %Πακέτο για εμφάνιση βιβλιογραφίας στα περιεχόμενα
\addto{\captionsenglish}{\renewcommand*{\contentsname}{\gr Περιεχόμενα}} %Αλλαγή του ονόματος των περιεχομένων
\renewcommand{\figurename}{\gr Εικόνα} %Αλλαγή του ονόματος του figure 
\usepackage{graphicx}                  %Πακέτο για την εισαγωγή εικόνων
\begin{document} 
\tableofcontents % εμφάνιση περιεχομένων
\newpage

\section{\gr τι είναι το \en Arch linux}

\begin{figure}[hbp!]
\centering
\includegraphics[natwidth=300bp,natheight=225bp]{archlinux.png}
\caption{\gr λογότυπο του \en Arch Linux}
\end{figure}

\gr Το \en Arch Linux\cite{archlinux} \gr είναι μια ανεξάρτητα ανεπτυγμένη \en i686/x86\_64 \gr διανομή γενικής χρήσεως, βασισμένη στο μοντέλο αναβαθμίσεων \en rolling-release\cite{rolling} \gr και αρκετά ευέλικτη ώστε να ταιριάζει σε οποιονδήποτε ρόλο. Η ανάπτυξή του εστιάζει στην απλοϊκότητα, στο μινιμαλισμό και τον ορθό κώδικα. Το \en Arch \gr εγκαθίσταται ως ένα βασικό σύστημα και ρυθμίζεται από το χρήστη ώστε να χρησιμοποιεί το ιδανικό του γραφικό περιβάλλον και να εγκαθιστά μόνο ότι βρίσκει επιθυμητό για τους μοναδικούς του σκοπούς\footnote{\en {https://en.wikipedia.org/wiki/Arch\_Linux}}. 

\pagebreak

\subsection{\gr Πλεονεκτήματά}
Το \en Arch \gr έχει ένα ελαφρύ περιβάλλον μετά την εγκατάσταση, (περιβάλλον γραμμής εντολών), ήδη συνταγμένο και βελτιστοποιημένο για \en i686/x86\_64 \gr αρχιτεκτονικές. Είναι γρήγορο, ελαφρύ, ευέλικτο και απλό. Η φιλοσοφία και η εφαρμογή του σχεδιασμού του, το κάνουν εύκολο στο να επεκταθεί και να προσαρμοστεί σε οποιοδήποτε τύπο μηχανήματος και αν χτίζετε-από ένα μινιμαλιστικό μηχάνημα κονσόλας, στα πιο επιβλητικά και πλούσια περιβάλλοντα χρήστη, διαθέσιμα. 

\subsubsection{\gr Διαχείριση πακέτων}

Το \en Arch \gr υποστηρίζεται από ένα εύκολο στη χρήση σύστημα δυαδικών πακέτων, το pacman, που επιτρέπει στο χρήστη την αναβάθμιση ολόκληρου του συστήματός σας με μία εντολή. Ο \en Pacman \gr είναι γραμμένος σε \en C \gr και είναι ελαφρύς και γρήγορος. Το \en Arch \gr επίσης χρησιμοποιεί ένα σύστημα παρόμοιο με τα \en ports \gr για χτίσιμο πακέτων \en (Arch Build System) \gr για να κάνει εύκολη την κατασκευή πακέτων \en (compile),\gr η οποία μπορεί επίσης να συγχρονιστεί με μία εντολή. Ολόκληρο το σύστημά μπορεί να ξανακτιστεί με μία εντολή. Το μοντέλο \en rolling release \gr επιτρέπει στο σύστημα την συνεχή και αδιάκοπη αναβάθμιση, χωρίς ανάγκη για επανεγκατάσταση.
Το \en Arch Linux \gr προσπαθεί να προσφέρει την σταθερότερη καινούρια έκδοση για το λογισμικό του. Οι χρήστες υποστηρίζουμε ένα βελτιωμένο πυρήνα πακέτων που έχει οριστεί για το βασικό\en i686 \gr και \en x86-64 \gr σύστημα, χιλιάδες πρόσθετα, υψηλής ποιότητας δυαδικά πακέτα που βρίσκονται ανάμεσα στα αποθετήρια χρηστών \en (Arch User Repositories, AUR) \gr, συνοδευόμενα \en PKGBUILD scripts\gr, για την κατασκευή \en (building) \gr και τη συσκευασία \en (packaging) \gr από την πηγή \en (source) .\gr Το \en Arch \gr στοχέυει στη διανομή προγραμμάτων που δεν τροποποιήθηκαν \en (vanilla software) \gr, δηλαδή τα πακέτα προσφέρονται όπως τα διανέμει εξ αρχής ο προγραμματιστής τους. Διορθώσεις προκύπτουν μόνο σε σπάνιες περιπτώσεις, ώστε να αποφευχθεί αστάθεια που μπορεί να προκύψει σε αυτό το μοντέλο αναβαθμίσεων.

\bibliographystyle{babunsrt}
\bibliography{cerebrux_3} 
\end{document}